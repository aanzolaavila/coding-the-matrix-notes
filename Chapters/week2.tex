\chapter{Week 2: The vector space}

\section{Linear combinations}
\begin{definition}
  An expression
  \begin{equation*}
    \alpha_1\vv_1 + \cdots + \alpha_n\vv_n
  \end{equation*}
  is a linear combination of the vectors $\vv_1, \ldots, \vv_n$.
  The scalars $\alpha_1, \ldots, \alpha_n$ are the coefficients of the linear combination.
\end{definition}

% ================================================================

\section{Span}
\begin{definition}[Span]
  The set of all linear combinations of some vectors $\vv_1, \ldots, \vv_n$ is called the span of these vectors. Written $\Span\{\vv_1,\ldots,\vv_n\}$.
\end{definition}

\begin{definition}
  Let $\sV$ be a set of vectors. If $\vv_1, \ldots, \vv_n$ are vectors such that $\sV = \Span\{ \vv_1, \ldots, \vv_n \}$, then
  \begin{itemize}
  \item We say $\{ \vv_1, \ldots, \vv_n \}$ is a generating set for $\sV$.
  \item We refer to the vectors $\vv_1, \ldots, \vv_n$ as generators for $\sV$.
  \end{itemize}
\end{definition}

If we use the vectors $[1, 0, 0]$, $[0, 1, 0]$, and $[0, 0, 1]$:
\begin{equation*}
  [x, y, z] = x[1, 0, 0] + y[0, 1, 0] + z[0, 0, 1]
\end{equation*}
These are called \emph{standard generators} for $\R^3$.

% ================================================================

\section{Geometry of sets of vectors}
Span of a single nonzero vector $\vv$:
\begin{equation*}
  \Span\{\vv\} = \{ \alpha\vv : \alpha\in\R \}
\end{equation*}
This is the line through the origin and $\vv$, One-dimensional. The span of the empty set: just the origin, Zero-dimensional.

One way to specify a plane can be with:
\begin{equation*}
  \{ (x, y, z) : ax + by + cz = 0 \}
\end{equation*}

Using dot-product, we could rewrite as
\begin{equation*}
  \{ [x, y, z] : [a, b, c] \cdot [x, y, z] = 0 \}
\end{equation*}

We can specify a line in three dimensions:
\begin{equation*}
  \{ [x, y, z] : \va_1 \cdot [x, y, z] = 0, \va_2 \cdot [x, y, z] = 0 \}
\end{equation*}

Two ways to represent a geometric object (line, plane, etc.) containing the origin:
\begin{itemize}
\item Span of some vectors.
\item Solution set of some system of linear equations with zero right hand sides.
\end{itemize}

What is common with these two representations? R: Subset $\fF^D$ satisfies three properties:
\begin{enumerate}
\item Subset contains the zero vector $\vzero$.
\item If subset contains $\vv$ then it contains $\alpha\vv$ for every scalar $\alpha$.
\item If subset contains $\vu$ and $\vv$ then it contains $\vu + \vv$.
\end{enumerate}

\begin{definition}[Vector space]
  Any subset $\sV$ of $\fF^{D}$ satisfying the three properties is called a vector space.
\end{definition}

\begin{definition}
  If $\sU$ is also a vector space and $\sU$ is a subset of $\sV$ then $\sU$ is called a subspace of $\sV$.
\end{definition}

\subsection{Convex hull}
\begin{definition}
  For vectors $\vv_1, \ldots, \vv_n$ over $\R$, a linear combination
  \begin{equation*}
    \alpha_1\vv_1, \ldots, \alpha_n\vv_n
  \end{equation*}
  is a convex combination if the coefficients are all nonnegative and they sum to $1$.
\end{definition}

% ================================================================

\section{Vector spaces}
To represent an object that doesn't contain the origin we sum a vector $\vc$ and redefine the definition as:
\begin{equation*}
  \{ \vc + \vv : \vv \in \sV \}
\end{equation*}
It can also be abbreviated as $\{ \vc + \sV \}$.

\begin{definition}
  A linear combination
  \begin{equation*}
    \alpha_1\vu_1 + \alpha_2\vu_2 + \cdots + \alpha_n\vu_n
  \end{equation*}
  where
  \begin{equation*}
    \alpha_1 + \alpha_2 + \cdots + \alpha_n = 1
  \end{equation*}
  is an affine combination.
\end{definition}

\begin{definition}
  The set of all affine combinations of vectors $\vu_1, \vu_2, \ldots, \vu_n$ is called the \emph{affine hull} of those vectors.
\end{definition}

Affine hull of $\vu_1, \vu_2, \ldots, \vu_n = \{ \vu_1 + \Span\{ \vu_2 - \vu_1, \ldots, \vu_n - \vu_1 \} \}$. This shows that the affine hull is an affine space.

In general, a geometric object can be expressed as the solution set of a system of linear equations.
\begin{equation*}
  \{ \vx : \va_1\cdot\vx = \beta_1, \cdots, \va_m\cdot\vx = \beta_m \}
\end{equation*}

Conversely, is the solution set an affine space?
Consider solution set of a contradictory system of equations, e.g. $1x = 1, 2x = 1$:
\begin{itemize}
\item Solution set is empty
\item but a vector space $\sV$ always contains the zero vector,
\item so an affine space $\vu_1 + \sV$ always contains at least one vector.
\end{itemize}
Turns out this is the only exception:
\begin{theorem}
  The solution set of a linear system is either empty or an affine space.
\end{theorem}

\begin{definition}
  A linear equation $\va\cdot\vx = 0$ with zero right-hand side is a homogeneous linear equation. A system of homogeneous linear equations is called a homogeneous linear system.
\end{definition}

\begin{lemma}
  Let $\vu_1$ be a solution to a linear system. Then, for any other vector $\vu_2$, $\vu_2$ is also a solution if and only if $\vu_2-\vu_1$ is a solution to the corresponding homogeneous linear system.
\end{lemma}

