\chapter{Week 3: The matrix}

\section{What is a matrix}
The traditional notion of a matrix: a two dimensional array
\begin{equation*}
  \mA = 
  \begin{bmatrix}
    a & b & c \\
    d & e & f
  \end{bmatrix}
\end{equation*}
where $\mA$ is a $2 \times 3$ matrix.

For a matrix $\mA$, the $i, j$ element of $\mA$ is the element in row $i$, column $j$. It is traditionally written as $\mA_{i, j}$

\begin{definition}
  For finite sets $\sR$ and $\sC$, and $\sR \times \sC$ matrix over field $\fF$ is a function from $\sR \times \sC$ to $\fF$.
\end{definition}

\begin{definition}
  $D \times D$ identity matrix is the matrix $\mathds{1}_D$ such that $\mathds{1}_D[k, k] = 1$ for all $k \in \{1, , \ldots, D\}$ and zero elsewhere.
\end{definition}

Usually we omit the subscript when $D$ is clear from the context.
Often letter $\mI$ is used instead of $\mathds{1}$

\subsection{Column space and row space}
One simple role for a matrix: packing together a bunch of columns or rows.
\begin{definition}
  Two vectors associated with a matrix $\mM$:
  \begin{itemize}
  \item Column space of $\mM = \Span\{\text{columns of }\mM\}$. Written $\Col \mM$.
  \item Row space of $\mM = \Span\{\text{rows of }\mM\}$. Written $\Row \mM$.
  \end{itemize}
\end{definition}

\subsection{Transpose}
Transpose swaps rows and columns. It is written as $\mA^{\top}$.
\begin{equation*}
  \mA = 
  \begin{bmatrix}
    a & b & c \\
    d & e & f
  \end{bmatrix}
\end{equation*}

\begin{equation*}
  \mA^{\top} = 
  \begin{bmatrix}
    a & d \\
    b & e \\
    c & f
  \end{bmatrix}
\end{equation*}

\subsection{Matrices as vectors}
A matrix can be interpreted as a vector:
\begin{itemize}
\item an $\sR \times \sS$ matrix is a function from $\sR \times \sS$ to $\fF$,
\item it can be interpreted as and $\sR \times \sS$--vector:
  \begin{itemize}
  \item scalar-vector multiplication
  \item vector addition
  \end{itemize}
\end{itemize}